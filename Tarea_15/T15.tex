\documentclass[spanish,notitlepage,letterpaper, 12pt]{article} % para articulo en castellano
\usepackage[ansinew]{inputenc} % Acepta caracteres en castellano
\usepackage[spanish]{babel} % silabea palabras castellanas
\usepackage{url}
\usepackage{cite}
\usepackage{amsmath}
\usepackage{amsfonts}
\usepackage{float}
\usepackage{amssymb}
\usepackage{hyperref}
\usepackage{graphicx}
\usepackage{geometry}      % See geometry.pdf to learn the layout options.
\geometry{letterpaper}                   % ... or a4paper or a5paper or ...
%\geometry{landscape}                % Activate for for rotated page geometry
%\usepackage[parfill]{parskip}    % Activate to begin paragraphs with an empty line rather than an indent
\usepackage{epstopdf}
\usepackage{fancyhdr} % encabezados y pies de pg
\usepackage{pgf,pgfarrows,pgfnodes}
\usepackage{graphicx}
\usepackage{subfig}
\usepackage{multirow}
\usepackage{mathtools}
\usepackage{color}
\usepackage{natbib}
\hypersetup{
    colorlinks,
    citecolor=violet,
    linkcolor=red,
    urlcolor=blue}
\spanishdecimal{.}

\renewcommand{\spanishtablename}{Cuadro}
%\renewcommand{\theenumi}{\alph{enumi}}

\pagestyle{fancy}
\chead{\bfseries Tarea 15}
\lhead{} % si se omite coloca el nombre de la seccion
\rhead{ }
\lfoot{\it }
\cfoot{}
\rfoot{\thepage}

\voffset = -0.25in
\textheight = 9.0in
\textwidth = 6.5in
\oddsidemargin = 0.in
\headheight = 20pt
\headwidth = 6.5in
\renewcommand{\headrulewidth}{0.5pt}
\renewcommand{\footrulewidth}{0,5pt}
\DeclareGraphicsRule{.tif}{png}{.png}{`convert #1 `dirname #1`/`basename #1 .tif`.png}

\begin{document}
\title{Tarea 15 \\ Propuestas proyecto}

\author{
\textbf{Erick Cervantes Mendieta} \\
\vspace{0.5cm}
\textnormal{Matr�cula: 2032430}\\
\textit{Modelos Probabilistas Aplicados}}
\date{15/12/2020}

\maketitle

%------------------------------------------------

\section{Nodos conectados}

En mi investigaci�n de tesis, se esta analizando una variante del problema de enrutamiento de veh�culos, en el cu�l, los veh�culos salen del dep�sito inicial y deben moverse a alg�n nodo en donde se requiera satisfacer la demanda de alg�n cliente, para nuestro caso el primer movimiento es hacia alg�n hotel, por lo que se desea analizar la frecuencia/probabilidad con la que los nodos asignados a los \emph{hoteles} est�n conectados en la ruta de alg�n veh�culo o no lo est�n en las diferentes soluciones encontradas, ya que esto permitir� ver la pertinencia de analizar o no un modelo en d�nde los hoteles sean tratados como dep�sitos iniciales.

\section{Nodos transbordo}

En la literatura podemos encontrar que el uso de nodos \emph{transbordo} mejora en varias ocasiones las soluciones �ptimas encontradas en problemas de enrutamiento de veh�culos con recogida y entrega, por lo que se desea saber si en la modelaci�n t�pica es conveniente o no agregar este tipo de restricci�n al problema, as�, al analizar algunas instancias modificando algunos par�metros, se espera observar cu�l es el comportamiento en la soluci�n (mejora o empeora), al implementar nodos transbordo en la modelaci�n.

\section{Dise�o de experimentos para hornear galletas}

Un negocio peque�o dedicado a la venta de galletas de mantequilla, desea saber los niveles �ptimos para hornear sus galletas, ya que ha decidido innovar y probar nuevos ingredientes y as� poder ofertar otro tipo de producto, algunos factores que se suponen afecta la calidad de sus galletas son: el tiempo de horneado, el tipo de horno a utilizar, el tipo de harina (trigo o almendra), la temperatura del horno, el grosor de las galletas. Por lo que es necesario determinar mediante un dise�o de experimentos, qu� factores son los que afectan m�s al proceso de horneado de un tipo de galleta y el nivel requerido para cada uno de ellos.

\section{Cartas de control para analizar los objetivos de calidad de una empresa}

Un ingeniero de calidad en una empresa cuenta con una base de datos, los cuales miden los objetivos de calidad por mes durante el a�o como lo es el nivel de inventario, la producci�n de los productos, nivel de desperdicios generados, la cantidad de clientes no conformes, y desea establecer, mediante el uso de cartas de control, los l�mites para que cada uno de estos objetivos se cumplan con lo establecido por la empresa.

%\bibliography{MiBiblio}
%\bibliographystyle{abbrv}

\end{document} 